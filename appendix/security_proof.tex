\section{Security Proof}

We assume an adversary attempting to subvert the security of our construction. Therefore, $\mathcal{A}$ may attempt to explore different attack vectors. For example, $\mathcal{A}$ may attempt to forge a proof of inclusion for the used Merkle tree, produce a zero knowledge proof forgery, randomly go offline in an attempt to disrupt the liveness of the system, or even even censor specific transactions from users. These represent different attack vectors that we model in this section.

\subsection{Safety}

To break the safety of the rollup system, $\mathcal{A}$ may target the soundness of the used zero-knowledge scheme to prove ownership of funds. This assumptions stems from the fact that the soundness of the zero-knowledge scheme guarantees with very high probability that any attempt to forge or modify a valid state will be detected, thus preserving the security of the system. 

\subsection*{Zero-Knowledge Proof Forgery}

\begin{theorem}
Given a zero-knowledge proof $\pi$, a statement $x$, and a set of public parameters $pp$ generated to provide a security parameter $\lambda$, the adversary $\mathcal{A}$ has a negligible probability of producing a zero-knowledge proof forgery, assuming the soundness property of the zero-knowledge scheme.
\end{theorem}

\begin{proof}(Sketch.)
We consider the soundness of the zero-knowledge scheme a critical property for ensuring the security of the proof. The soundness property guarantees that an adversary $\mathcal{A}$ cannot produce a valid zero-knowledge proof unless they possess the correct witness.
To break the soundness property, $\mathcal{A}$ must find a witness $w'$ that makes the verifier accept an invalid proof $\pi'$ generated from $\mathsf{Prove}(pp, x, w')$. However, the soundness property ensures that the probability of $\mathcal{A}$ successfully executing this attack is negligible, typically bounded by $2^{-k}$ where $k$ represents the knowledge error.

Therefore, as long as the zero-knowledge scheme is instantiated with appropriate parameters and exhibits the soundness property, the probability of an adversary producing a zero-knowledge proof forgery is negligible.

Thus, based on the assumption of soundness and the negligible probability of forging a zero-knowledge proof, we can conclude that the zero-knowledge scheme provides the desired security against proof forgery attempts.
\end{proof}


\subsection*{Commitment Scheme}

To break the safety of the rollup system, $\mathcal{A}$ may target the security properties of the used commitment scheme, which ensures the integrity of each new state. 

\begin{theorem}
    Given a commitment $\mathcal{C}$ and a transaction $\mathsf{tx}$ such that \\$\mathsf{Commit(tx)} \rightarrow \mathcal{C}$, $\mathcal{A}$ has negligible probability of producing a $\mathsf{tx'} \neq \mathsf{tx}$ such that $\mathsf{Commit(tx')} = \mathsf{Commit(tx)}$, if the used commitment scheme is binding. 
\end{theorem}

\begin{proof}(Sketch.)
We aim to prove that, assuming a binding property of the used commitment scheme, the probability of an adversary $\mathcal{A}$ producing a different value that matches the commitment value is negligible.

The binding property ensures that it is not computationally feasible to manipulate the opening phase and use a different value as it results in the commitment opening to a different message. 

Consider the scenario where $\mathcal{A}$ attempts to produce a malicious value for a given commitment $\mathcal{C}$ to a transaction $\mathsf{tx}$. To succeed, $\mathcal{A}$ must find a rogue transaction $\mathsf{tx'} \neq \mathsf{tx}$ such that $\mathsf{Commit(tx')} = \mathsf{Commit(tx)}$. The binding property guarantees that the probability of finding such a transaction is negligible. 
\end{proof}

\subsection{Liveness}

To break the liveness property of the system, the adversary may attempt to go offline over extended periods of time or by censoring transactions from specific users. We now show that these attacks do not compromise the liveness property of the system. 

\begin{theorem}
In a rollup system with a designated aggregator responsible for submitting batch updates to the underlying layer 1, if a malicious aggregator attempts to disrupt liveness by going offline, the system can maintain liveness as long as there exists at least one honest participant in the system who can assume the role of the aggregator.
\end{theorem}

\begin{proof}(Sketch.)
We aim to prove that in the given rollup system, liveness can be sustained even if a malicious aggregator goes offline, as long as there exists at least one honest participant in the system who can seamlessly transition to the role of the aggregator.

Let us consider a scenario where a malicious aggregator intentionally goes offline, disrupting the regular batch update process. Due to the decentralized nature of the rollup system, any honest participant can readily assume the role of the aggregator.

Since the rollup system does not depend on any specific entity as the aggregator, the ability to transition the role to an honest participant ensures the continuity of transaction processing and updates. The honest participant, upon assuming the aggregator role, can effectively submit batch updates to the underlying layer 1, thereby maintaining the liveness property of the system.

Thus, we can conclude that in the given rollup system, liveness can be maintained despite the malicious aggregator going offline, as long as there exists at least one honest participant who can assume the role of the aggregator.
\end{proof}

\begin{theorem}
In a rollup system with a designated aggregator responsible for submitting batch updates to the underlying layer 1, if a malicious aggregator attempts to censor transactions from users, the system can overcome censorship and maintain liveness if one or more honest party assumes the role of the aggregator.
\end{theorem}

\begin{proof}(Sketch.)
We aim to prove that in a rollup system where an aggregator is responsible for submitting batch updates to the underlying layer 1, if a malicious aggregator attempts to censor transactions from users, the system can overcome censorship as long as each of these censored users can assume the role of the aggregator.

Consider a scenario where a malicious aggregator attempts to censor transactions from certain users by intentionally excluding their transactions from the batch updates. However, the decentralized design of the rollup system empowers users to become aggregators themselves.

In this case, if a user perceives censorship or exclusion of their transactions by the aggregator, they can opt to become an aggregator and directly submit batch updates to the underlying layer 1. By taking over the aggregator role, the user-turned-aggregator ensures that their transactions are included in the batch updates.

The ability of users to bypass the malicious aggregator and become aggregators themselves provides a mechanism to overcome censorship within the system, which ensures that transactions from users are not unduly suppressed or excluded, maintaining the desired liveness property.

Therefore, we can conclude that in such a rollup system, even if a malicious aggregator attempts to censor transactions from users, the system can overcome censorship and maintain liveness as long as there exists at least one honest participant who can assume the role of the aggregator.
\end{proof}