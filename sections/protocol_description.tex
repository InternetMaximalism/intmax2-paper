
In Intmax2, we leverage the use of recursive zero-knowledge proofs to shift the computationally expensive part of the zk-rollups to the user side. As a result, we obtain a more lightweight aggregator (or validator). A side effect of this design decision is that the aggregator role is easily decentralizable. 

In our proposed protocol, there are two roles running simultaneously. The role of a user and the role of the aggregator. We highlight, however, that since the aggregator is easily decentralizable, a user can become an aggregator, thus ensuring the continuous operation and liveness of the system.

\paragraph{Aggregator Role.}
In Intmax2, the aggregator plays a crucial role in the transaction processing and updating of the underlying layer 1. The aggregator is responsible for collecting and batching user transactions into batches, which are subsequently posted to the underlying layer 1. By doing so, the aggregator facilitates the efficient and secure execution of transactions in the system.

\paragraph{User Role.}
In Intmax2, users are involved in transacting funds. To do so, users must perform a sequence of steps. First, a user must register a (BLS) public key. Upon registration, to start transacting, a user must deposit funds into their account. Once a successful deposit occurs, a user is able to transfer funds to other users. Finally, users can perform withdrawals. 

\subsection{Registration}
    \begin{Protocol*}[h]
\begin{mdframed}

%\footnotesize
\fontsize{10pt}{2cm}

% == 

\begin{center}
    \textbf{Registration}
\end{center}

\smallbreak

Before a user can transact on the rollup, user Alice must register a BLS public key. To do so, user Alice performs the following steps:

\begin{itemize}
  \setlength\itemsep{0.25em}

    \item Alice generates a BLS secret key $x \xleftarrow[]{R} \mathbb{Z}_{q}$
    
    \item Alice obtains the corresponding public key $pk \xleftarrow[]{} g_{1}^{x} \in \mathbb{G}_{1}$

    \item Alice produces a signature $\sigma \xleftarrow[]{}\mathcal{H}(m)^{x} \in \mathbb{G}_{0}$, where \(m\) is the message ``I am registering the BLS public key \emph{pk} on Intmax2", which is a registration message exclusive to Alice and is cryptographically binding.

    \item Alice outputs the following registration block: $(pk, \sigma)$

\end{itemize}

Upon successful registration, an L2 address is assigned to Alice.
\smallbreak
In this step, the signature proves that each user knows the private key corresponding to their public key, preventing the rogue key attack on BLS signatures. When a user registers a new account, the account is given an L2 address, which is an integer that increments for each new account.


\normalsize	
\end{mdframed}
\caption{Registration Protocol.
\label{alg:reg}}
\end{Protocol*}

\subsection{Deposit}
    \begin{Protocol*}[!h]
\begin{mdframed}

%\footnotesize
\fontsize{10pt}{2cm}

\begin{center}
    \textbf{Depositing to rollup}
\end{center}
\smallbreak
In order to transact on the rollup, users must have a token balance on the rollup. To have such a balance, the user can either receive funds from another L2 user or deposit the funds themselves. We now describe the setting where Alice performs her own deposit of funds. To do so, user Alice performs the following steps:

\begin{itemize}
  \setlength\itemsep{0.25em}

    \item Alice creates a deposit block containing the destination L2 address and the amount of each token to be deposited. 

    \item Alice submits the deposit block to the rollup smart contract together with the specified amount of each token. 
\end{itemize}


In this step, the destination L2 address does not necessarily have to belong to Alice, as she may be attempting to deposit funds into someone else's account. 

\end{mdframed}
\caption{Deposit Protocol.\label{alg:deposit}}
\end{Protocol*}

\clearpage
\subsection{Transfer}
    \begin{Protocol*}[ht!]
\begin{mdframed}

\textbf{Transfer}
\smallbreak
To perform a transfer of funds, user Alice performs the following steps:
\vspace{-1mm}
\begin{itemize}
    \setlength\itemsep{0.15em}
    
    \item Alice creates a transaction $\mathsf{tx}$ specifying the desired transfer of funds

    \item Alice sends the transaction to the aggregator
    
\end{itemize}

The aggregator, to process a transaction, performs the following steps:
\vspace{-1mm}
\begin{itemize}
    \setlength\itemsep{0.15em}
    
    \item The aggregator collects a set of received transactions, and produces a merkle tree containing all the corresponding transactions. 

    \item The aggregator individually produces the merkle proofs of inclusion for each transaction in the included set
    
    \item The aggregator sends each merkle proof of inclusion to the corresponding user performing the transaction
\end{itemize}

To confirm the transaction, user Alice performs the following steps:
\vspace{-1mm}
\begin{itemize}
    \setlength\itemsep{0.15em}
    
    \item Alice checks that the merkle proof of inclusion is correct.

    \item Alice signs the merkle root of the tree. This is to confirm that the tree contains her transaction.
    
    \item Alice sends the signed merkle root to the aggregator. 
\end{itemize}

The aggregator, to finalize a set of transactions, performs the following steps:
\vspace{-1mm}
\begin{itemize}
    \setlength\itemsep{0.15em}
    
    \item The aggregator collects the set of received signatures on the produced merkle root.

    \item The aggregator aggregates the received signatures into a single aggregated signature and obtains a finalized set of transactions, which comprises the merkle root and the aggregated signature on such root. This finalized set of transactions is referred to as the transfer block since it contains a block of transfers (or transactions). 
    
    \item The aggregator submits the transfer block to the rollup smart contract. 
\end{itemize}

We highlight that the transfer block does not need to contain the actual transactions as the proofs for inclusion of the transactions have been shared with each individual user. Therefore, only the merkle root and the signature are posted on the underlying L1, thus resulting in an approach that is very data-efficient, unlike traditional zk-rollup approaches.

\end{mdframed}
\caption{Transfer Protocol.\label{alg:transfering}}
\end{Protocol*}

\subsection{Withdrawal}
    \begin{Protocol*}[h!]
\begin{mdframed}

\fontsize{10pt}{2cm}

\textbf{Withdrawing funds} 
\smallbreak

To withdraw funds, user Alice performs the following steps:


\begin{itemize}
    \setlength\itemsep{0.15em}

    \item Alice sends in a transfer block the desired amount to be withdrawn from her L2 account to the rollup account representing her L1 account

    \item Alice produces a zero knowledge proof $P$ that proves that the rollup account representing her L1 account has a certain balance at a previous index of the rollup: $\mathsf{Prove}(pp, x, w) \rightarrow P$.

    \item Alice submits the balance proof \(P\) to the withdrawal function in the rollup contract.
\end{itemize}

Upon receiving the proof, the withdrawal function performs the following steps:

\begin{itemize}
    \setlength\itemsep{0.15em}

    \item Withdrawal function verifies that the zero-knowledge proof verification outputs true: $\mathsf{Verify}(pp, x, P) \rightarrow \mathsf{Accept}$

    \item Checks the provided rollup hash is in the list of previous rollup hashes.

    \item Transfers to the L1 account (on L1) the difference between the proven balance of the rollup account representing her L1 address and the amount that has previously been withdrawn to the L1 address, and updates the total amount withdrawn to the L1 address accordingly in the contract storage.
\end{itemize}

It is important to note that if a specific use case allows for the constant use of the funds in the rollup, then a user does not necessarily have to withdraw funds from the rollup and can constantly use the existing funds and subsequently deposit (or receive) funds on an ongoing basis. 

\normalsize	
\end{mdframed}
\caption{Withdrawal Protocol.\label{box:withdrawal}}
\end{Protocol*}

    